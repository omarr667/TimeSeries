\usepackage{booktabs}
\usepackage{xcolor} 
\definecolor{gray97}{gray}{.97}
\definecolor{gray97}{gray}{.97}
\definecolor{airforceblue}{rgb}{0.36, 0.54, 0.66}
\definecolor{Mahogany}{rgb}{0.75, 0.25, 0.0}
\definecolor{Olive}{rgb}{0.5, 0.5, 0.0}
\definecolor{colortitulo}{RGB}{219,68,14} % 
\definecolor{colordominante}{RGB}{243,102,25}
\definecolor{colordominanteF}{RGB}{219,68,14}
\definecolor{colordominanteD}{RGB}{137,46,55}
\definecolor{mostaza}{RGB}{231,196,25}
\definecolor{amarilloM}{RGB}{248,199,90}
\definecolor{amarilloD}{RGB}{251,237,121}
\definecolor{grisamarillo}{RGB}{248,248,245} 
\definecolor{azulF}{rgb}{.0,.0,.3}
\definecolor{grisD}{rgb}{.3,.3,.3}
\definecolor{grisF}{rgb}{.82,.82,.82}
\definecolor{miverde}{RGB}{44,162,67}
\newcommand{\verde}{\color{miverde}}

\usepackage{listings}
% Puede usar lstlisting|texto| para código en el texto
\lstset{ 
	language=R,                     % the language of the code
	basicstyle=\footnotesize\ttfamily, % the size of the fonts that are used for the code
	%numbers=left,                   % where to put the line-numbers
	%numberstyle=\tiny\color{Blue},  % the style that is used for the line-numbers
	%stepnumber=1,                   % the step between two line-numbers. If it is 1, each line
	% will be numbered
	%numbersep=5pt,                  % how far the line-numbers are from the code
	frame=single,
	framextopmargin=3pt,
	framexbottommargin=3pt,
	framexleftmargin=0.4cm,
	backgroundcolor=\color{gray97},
	showspaces=false,               % show spaces adding particular underscores
	showstringspaces=false,         % underline spaces within strings
	showtabs=false,                 % show tabs within strings adding particular underscores
	frame=single,                   % adds a frame around the code
	rulecolor=\color{black},        % if not set, the frame-color may be changed on line-breaks within not-black text (e.g. commens (green here))
	tabsize=2,                      % sets default tabsize to 2 spaces
	captionpos=b,                   % sets the caption-position to bottom
	breaklines=true,                % sets automatic line breaking
	breakatwhitespace=false,        % sets if automatic breaks should only happen at whitespace
	keywordstyle=\color{Mahogany},      % keyword style
	commentstyle=\color{airforceblue},   % comment style
	stringstyle=\color{Olive}      % string literal style
} 


%\usepackage[width=17cm,height=23cm]{geometry}
%\geometry{bindingoffset=1.2cm}
\textheight=22cm
\textwidth=15.5cm
\topmargin=-1cm
\oddsidemargin = -0cm %margen izquierdo, página impar
\evensidemargin = -0cm %margen izquierdo, página par

\newcommand{\helv}{\fontfamily{phv}\fontsize{9}{11}\selectfont}


\usepackage{fancyhdr}
\pagestyle{fancy}

\renewcommand{\chaptermark}[1]{\markboth{#1}{}}
\renewcommand{\sectionmark}[1]{\markright{\thesection\ #1}}
\fancyhf{} % borra cabecera y pie actuales
\fancyhead[LE,RO]{\helv\thepage} %Left Even page - Right Odd page
\fancyhead[LO]{\helv\rightmark}
\fancyhead[RE]{\helv\leftmark}
%\renewcommand{\headrulewidth}{0pt} % Sin raya. Con raya?: cambiar {0} por {0.5pt}
%\renewcommand{\footrulewidth}{0pt}
\renewcommand{\headrulewidth}{0.5pt} % grosor 0.5pt
\addtolength{\headheight}{0.5pt} % espacio para la raya
\fancyheadoffset{0 cm} %Controla el tamaño de la línea



\usepackage{mdframed}

\usepackage{amsthm}
\newtheorem{theorem}{Teorema}[chapter]
\newtheorem{lemma}{Lema}[chapter]
\newtheorem{corollary}{Corolario}[chapter]
\newtheorem{proposition}{Proposición}[chapter]
\newtheorem{conjecture}{Conjecture}[chapter]

\newtheorem{del}{Definición}[chapter]
\newenvironment{definition}
{\begin{mdframed}[backgroundcolor=grisF, rightline=false,leftline=false,topline=false, bottomline=false]\begin{del}}
		{\end{del}\end{mdframed}}

\newtheorem{example}{Ejemplo}[chapter]
\newtheorem{exercise}{Ejercicio}[chapter]
\newtheorem{hypothesis}{Hypothesis}[chapter]
\theoremstyle{remark}
\newtheorem*{remark}{Nota: }
\newtheorem*{solution}{Solución}

\parskip=0.4cm
\parindent=3mm